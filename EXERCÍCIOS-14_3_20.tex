\PassOptionsToPackage{unicode=true}{hyperref} % options for packages loaded elsewhere
\PassOptionsToPackage{hyphens}{url}
%
\documentclass[]{article}
\usepackage{lmodern}
\usepackage{amssymb,amsmath}
\usepackage{ifxetex,ifluatex}
\usepackage{fixltx2e} % provides \textsubscript
\ifnum 0\ifxetex 1\fi\ifluatex 1\fi=0 % if pdftex
  \usepackage[T1]{fontenc}
  \usepackage[utf8]{inputenc}
  \usepackage{textcomp} % provides euro and other symbols
\else % if luatex or xelatex
  \usepackage{unicode-math}
  \defaultfontfeatures{Ligatures=TeX,Scale=MatchLowercase}
\fi
% use upquote if available, for straight quotes in verbatim environments
\IfFileExists{upquote.sty}{\usepackage{upquote}}{}
% use microtype if available
\IfFileExists{microtype.sty}{%
\usepackage[]{microtype}
\UseMicrotypeSet[protrusion]{basicmath} % disable protrusion for tt fonts
}{}
\IfFileExists{parskip.sty}{%
\usepackage{parskip}
}{% else
\setlength{\parindent}{0pt}
\setlength{\parskip}{6pt plus 2pt minus 1pt}
}
\usepackage{hyperref}
\hypersetup{
            pdftitle={EXERCÍCIOS-14\_3\_20.R},
            pdfauthor={rstudio-user},
            pdfborder={0 0 0},
            breaklinks=true}
\urlstyle{same}  % don't use monospace font for urls
\usepackage[margin=1in]{geometry}
\usepackage{color}
\usepackage{fancyvrb}
\newcommand{\VerbBar}{|}
\newcommand{\VERB}{\Verb[commandchars=\\\{\}]}
\DefineVerbatimEnvironment{Highlighting}{Verbatim}{commandchars=\\\{\}}
% Add ',fontsize=\small' for more characters per line
\usepackage{framed}
\definecolor{shadecolor}{RGB}{248,248,248}
\newenvironment{Shaded}{\begin{snugshade}}{\end{snugshade}}
\newcommand{\AlertTok}[1]{\textcolor[rgb]{0.94,0.16,0.16}{#1}}
\newcommand{\AnnotationTok}[1]{\textcolor[rgb]{0.56,0.35,0.01}{\textbf{\textit{#1}}}}
\newcommand{\AttributeTok}[1]{\textcolor[rgb]{0.77,0.63,0.00}{#1}}
\newcommand{\BaseNTok}[1]{\textcolor[rgb]{0.00,0.00,0.81}{#1}}
\newcommand{\BuiltInTok}[1]{#1}
\newcommand{\CharTok}[1]{\textcolor[rgb]{0.31,0.60,0.02}{#1}}
\newcommand{\CommentTok}[1]{\textcolor[rgb]{0.56,0.35,0.01}{\textit{#1}}}
\newcommand{\CommentVarTok}[1]{\textcolor[rgb]{0.56,0.35,0.01}{\textbf{\textit{#1}}}}
\newcommand{\ConstantTok}[1]{\textcolor[rgb]{0.00,0.00,0.00}{#1}}
\newcommand{\ControlFlowTok}[1]{\textcolor[rgb]{0.13,0.29,0.53}{\textbf{#1}}}
\newcommand{\DataTypeTok}[1]{\textcolor[rgb]{0.13,0.29,0.53}{#1}}
\newcommand{\DecValTok}[1]{\textcolor[rgb]{0.00,0.00,0.81}{#1}}
\newcommand{\DocumentationTok}[1]{\textcolor[rgb]{0.56,0.35,0.01}{\textbf{\textit{#1}}}}
\newcommand{\ErrorTok}[1]{\textcolor[rgb]{0.64,0.00,0.00}{\textbf{#1}}}
\newcommand{\ExtensionTok}[1]{#1}
\newcommand{\FloatTok}[1]{\textcolor[rgb]{0.00,0.00,0.81}{#1}}
\newcommand{\FunctionTok}[1]{\textcolor[rgb]{0.00,0.00,0.00}{#1}}
\newcommand{\ImportTok}[1]{#1}
\newcommand{\InformationTok}[1]{\textcolor[rgb]{0.56,0.35,0.01}{\textbf{\textit{#1}}}}
\newcommand{\KeywordTok}[1]{\textcolor[rgb]{0.13,0.29,0.53}{\textbf{#1}}}
\newcommand{\NormalTok}[1]{#1}
\newcommand{\OperatorTok}[1]{\textcolor[rgb]{0.81,0.36,0.00}{\textbf{#1}}}
\newcommand{\OtherTok}[1]{\textcolor[rgb]{0.56,0.35,0.01}{#1}}
\newcommand{\PreprocessorTok}[1]{\textcolor[rgb]{0.56,0.35,0.01}{\textit{#1}}}
\newcommand{\RegionMarkerTok}[1]{#1}
\newcommand{\SpecialCharTok}[1]{\textcolor[rgb]{0.00,0.00,0.00}{#1}}
\newcommand{\SpecialStringTok}[1]{\textcolor[rgb]{0.31,0.60,0.02}{#1}}
\newcommand{\StringTok}[1]{\textcolor[rgb]{0.31,0.60,0.02}{#1}}
\newcommand{\VariableTok}[1]{\textcolor[rgb]{0.00,0.00,0.00}{#1}}
\newcommand{\VerbatimStringTok}[1]{\textcolor[rgb]{0.31,0.60,0.02}{#1}}
\newcommand{\WarningTok}[1]{\textcolor[rgb]{0.56,0.35,0.01}{\textbf{\textit{#1}}}}
\usepackage{graphicx,grffile}
\makeatletter
\def\maxwidth{\ifdim\Gin@nat@width>\linewidth\linewidth\else\Gin@nat@width\fi}
\def\maxheight{\ifdim\Gin@nat@height>\textheight\textheight\else\Gin@nat@height\fi}
\makeatother
% Scale images if necessary, so that they will not overflow the page
% margins by default, and it is still possible to overwrite the defaults
% using explicit options in \includegraphics[width, height, ...]{}
\setkeys{Gin}{width=\maxwidth,height=\maxheight,keepaspectratio}
\setlength{\emergencystretch}{3em}  % prevent overfull lines
\providecommand{\tightlist}{%
  \setlength{\itemsep}{0pt}\setlength{\parskip}{0pt}}
\setcounter{secnumdepth}{0}
% Redefines (sub)paragraphs to behave more like sections
\ifx\paragraph\undefined\else
\let\oldparagraph\paragraph
\renewcommand{\paragraph}[1]{\oldparagraph{#1}\mbox{}}
\fi
\ifx\subparagraph\undefined\else
\let\oldsubparagraph\subparagraph
\renewcommand{\subparagraph}[1]{\oldsubparagraph{#1}\mbox{}}
\fi

% set default figure placement to htbp
\makeatletter
\def\fps@figure{htbp}
\makeatother


\title{EXERCÍCIOS-14\_3\_20.R}
\author{rstudio-user}
\date{2020-03-14}

\begin{document}
\maketitle

\begin{Shaded}
\begin{Highlighting}[]
\CommentTok{# 1) Cite 4 principais tipos de objetos do R e explique cada um deles.}

  \CommentTok{# Vetor (vector): uma dos tipos mais usados e que permite o armazenagem de múltiplos}
  \CommentTok{# valores do mesmo tipo;}

  \CommentTok{# Matriz (matrix): tipo de dados que armazena valores em formato linha/coluna;}

  \CommentTok{# Array: semelhante ao tipo Matriz, entretanto, permite a definição de múltiplas dimensões;}

  \CommentTok{# Data frame (data.frame): é semelhante a uma matriz, mas permite utilizar tipos de dados}
  \CommentTok{# diferentes. É um dos tipos mais utilizados para análises devido as funções que possuem}
  \CommentTok{# e facilitam a análise;}

\CommentTok{# 2) Qual a vantagem de guardarmos informação categórica como fatores em vez de usarmos strings?}

  \CommentTok{# Fatores (factor): permite trabalhar de forma adequada com dados do tipo categórico, pois}
  \CommentTok{# o R possui funções para este tipo de dados que facilitam como, por exemplo,}
  \CommentTok{# identificar os níveis de categoria (levels);}

\CommentTok{# 3) Qual a principal característica de um data.frame?}

  \CommentTok{# No data.frame, cada coluna pode ter um tipo de dados diferente;}

\CommentTok{# 4) Monte um data.frame chamado macac, com os dados obtidos da reserva A e B, para macacos e}
\CommentTok{# quantidade de árvores frutíferas, de acordo com os vetores abaixo e responda as questões:}
\CommentTok{# Reserva: A,A,A,A,A,A,A,A,A,A,B,B,B,B,B,B,B,B,B,B}
\CommentTok{# Macacos: 22,28,37,34,13,24,39,5,33,32,7,15,12,14,4,14,16,60,13,16}
\CommentTok{# Frutíferas: 25,26,40,30,10,20,35,8,35,28,6,17,18,11,6,15,20,16,12,15}

\NormalTok{Reserva <-}\StringTok{ }\KeywordTok{c}\NormalTok{(}\StringTok{"A"}\NormalTok{,}\StringTok{"A"}\NormalTok{,}\StringTok{"A"}\NormalTok{,}\StringTok{"A"}\NormalTok{,}\StringTok{"A"}\NormalTok{,}\StringTok{"A"}\NormalTok{,}\StringTok{"A"}\NormalTok{,}\StringTok{"A"}\NormalTok{,}\StringTok{"A"}\NormalTok{,}\StringTok{"A"}\NormalTok{,}\StringTok{"B"}\NormalTok{,}\StringTok{"B"}\NormalTok{,}\StringTok{"B"}\NormalTok{,}\StringTok{"B"}\NormalTok{,}\StringTok{"B"}\NormalTok{,}\StringTok{"B"}\NormalTok{,}\StringTok{"B"}\NormalTok{,}\StringTok{"B"}\NormalTok{,}\StringTok{"B"}\NormalTok{,}\StringTok{"B"}\NormalTok{)}
\NormalTok{Macacos <-}\StringTok{ }\KeywordTok{c}\NormalTok{(}\DecValTok{22}\NormalTok{,}\DecValTok{28}\NormalTok{,}\DecValTok{37}\NormalTok{,}\DecValTok{34}\NormalTok{,}\DecValTok{13}\NormalTok{,}\DecValTok{24}\NormalTok{,}\DecValTok{39}\NormalTok{,}\DecValTok{5}\NormalTok{,}\DecValTok{33}\NormalTok{,}\DecValTok{32}\NormalTok{,}\DecValTok{7}\NormalTok{,}\DecValTok{15}\NormalTok{,}\DecValTok{12}\NormalTok{,}\DecValTok{14}\NormalTok{,}\DecValTok{4}\NormalTok{,}\DecValTok{14}\NormalTok{,}\DecValTok{16}\NormalTok{,}\DecValTok{60}\NormalTok{,}\DecValTok{13}\NormalTok{,}\DecValTok{16}\NormalTok{)}
\NormalTok{Frutíferas <-}\StringTok{ }\KeywordTok{c}\NormalTok{(}\DecValTok{25}\NormalTok{,}\DecValTok{26}\NormalTok{,}\DecValTok{40}\NormalTok{,}\DecValTok{30}\NormalTok{,}\DecValTok{10}\NormalTok{,}\DecValTok{20}\NormalTok{,}\DecValTok{35}\NormalTok{,}\DecValTok{8}\NormalTok{,}\DecValTok{35}\NormalTok{,}\DecValTok{28}\NormalTok{,}\DecValTok{6}\NormalTok{,}\DecValTok{17}\NormalTok{,}\DecValTok{18}\NormalTok{,}\DecValTok{11}\NormalTok{,}\DecValTok{6}\NormalTok{,}\DecValTok{15}\NormalTok{,}\DecValTok{20}\NormalTok{,}\DecValTok{16}\NormalTok{,}\DecValTok{12}\NormalTok{,}\DecValTok{15}\NormalTok{)}
\NormalTok{macac <-}\StringTok{ }\KeywordTok{data.frame}\NormalTok{(Reserva, Macacos, Frutíferas)}
\NormalTok{macac}
\end{Highlighting}
\end{Shaded}

\begin{verbatim}
##    Reserva Macacos Frutíferas
## 1        A      22         25
## 2        A      28         26
## 3        A      37         40
## 4        A      34         30
## 5        A      13         10
## 6        A      24         20
## 7        A      39         35
## 8        A       5          8
## 9        A      33         35
## 10       A      32         28
## 11       B       7          6
## 12       B      15         17
## 13       B      12         18
## 14       B      14         11
## 15       B       4          6
## 16       B      14         15
## 17       B      16         20
## 18       B      60         16
## 19       B      13         12
## 20       B      16         15
\end{verbatim}

\begin{Shaded}
\begin{Highlighting}[]
\CommentTok{# a) Verifique se a 1a coluna é um fator um caractere.}
\KeywordTok{is.factor}\NormalTok{(macac[,}\DecValTok{1}\NormalTok{])}
\end{Highlighting}
\end{Shaded}

\begin{verbatim}
## [1] TRUE
\end{verbatim}

\begin{Shaded}
\begin{Highlighting}[]
\KeywordTok{is.character}\NormalTok{(macac[,}\DecValTok{1}\NormalTok{])}
\end{Highlighting}
\end{Shaded}

\begin{verbatim}
## [1] FALSE
\end{verbatim}

\begin{Shaded}
\begin{Highlighting}[]
\CommentTok{# b) Caso a 1a coluna seja um fator, transforme em caracteres.}
\NormalTok{macac[,}\DecValTok{1}\NormalTok{] <-}\StringTok{ }\KeywordTok{as.character}\NormalTok{(macac[,}\DecValTok{1}\NormalTok{])}
\KeywordTok{is.character}\NormalTok{(macac[,}\DecValTok{1}\NormalTok{])}
\end{Highlighting}
\end{Shaded}

\begin{verbatim}
## [1] TRUE
\end{verbatim}

\begin{Shaded}
\begin{Highlighting}[]
\CommentTok{# c) Confira agora se a 1a coluna é um fator ou caractere através do comando mode e fazendo a}
\CommentTok{# pergunta através do is.}
\KeywordTok{mode}\NormalTok{(macac[,}\DecValTok{1}\NormalTok{])}
\end{Highlighting}
\end{Shaded}

\begin{verbatim}
## [1] "character"
\end{verbatim}

\begin{Shaded}
\begin{Highlighting}[]
\KeywordTok{is.factor}\NormalTok{(macac[,}\DecValTok{1}\NormalTok{])}
\end{Highlighting}
\end{Shaded}

\begin{verbatim}
## [1] FALSE
\end{verbatim}

\begin{Shaded}
\begin{Highlighting}[]
\KeywordTok{is.character}\NormalTok{(macac[,}\DecValTok{1}\NormalTok{])}
\end{Highlighting}
\end{Shaded}

\begin{verbatim}
## [1] TRUE
\end{verbatim}

\begin{Shaded}
\begin{Highlighting}[]
\CommentTok{# d) Acesse a coluna Macacos.}
\NormalTok{macac}\OperatorTok{$}\NormalTok{Macacos}
\end{Highlighting}
\end{Shaded}

\begin{verbatim}
##  [1] 22 28 37 34 13 24 39  5 33 32  7 15 12 14  4 14 16 60 13 16
\end{verbatim}

\begin{Shaded}
\begin{Highlighting}[]
\CommentTok{# e) Localize o 12o elemento de Macacos.}
\NormalTok{macac}\OperatorTok{$}\NormalTok{Macacos[}\DecValTok{12}\NormalTok{]}
\end{Highlighting}
\end{Shaded}

\begin{verbatim}
## [1] 15
\end{verbatim}

\begin{Shaded}
\begin{Highlighting}[]
\CommentTok{# f) Adicione uma coluna chamada Mortes, com os valores 2,7,1,2,7,4,2,4,3,9,6,6,4,1,3,1,7,2,1,8.}
\NormalTok{Mortes <-}\StringTok{ }\KeywordTok{c}\NormalTok{(}\DecValTok{2}\NormalTok{,}\DecValTok{7}\NormalTok{,}\DecValTok{1}\NormalTok{,}\DecValTok{2}\NormalTok{,}\DecValTok{7}\NormalTok{,}\DecValTok{4}\NormalTok{,}\DecValTok{2}\NormalTok{,}\DecValTok{4}\NormalTok{,}\DecValTok{3}\NormalTok{,}\DecValTok{9}\NormalTok{,}\DecValTok{6}\NormalTok{,}\DecValTok{6}\NormalTok{,}\DecValTok{4}\NormalTok{,}\DecValTok{1}\NormalTok{,}\DecValTok{3}\NormalTok{,}\DecValTok{1}\NormalTok{,}\DecValTok{7}\NormalTok{,}\DecValTok{2}\NormalTok{,}\DecValTok{1}\NormalTok{,}\DecValTok{8}\NormalTok{)}
\NormalTok{macac <-}\StringTok{ }\KeywordTok{cbind}\NormalTok{(macac, Mortes)}
\NormalTok{macac}
\end{Highlighting}
\end{Shaded}

\begin{verbatim}
##    Reserva Macacos Frutíferas Mortes
## 1        A      22         25      2
## 2        A      28         26      7
## 3        A      37         40      1
## 4        A      34         30      2
## 5        A      13         10      7
## 6        A      24         20      4
## 7        A      39         35      2
## 8        A       5          8      4
## 9        A      33         35      3
## 10       A      32         28      9
## 11       B       7          6      6
## 12       B      15         17      6
## 13       B      12         18      4
## 14       B      14         11      1
## 15       B       4          6      3
## 16       B      14         15      1
## 17       B      16         20      7
## 18       B      60         16      2
## 19       B      13         12      1
## 20       B      16         15      8
\end{verbatim}

\begin{Shaded}
\begin{Highlighting}[]
\CommentTok{# g) Exclua a coluna Frutíferas.}
\NormalTok{macac <-}\StringTok{ }\NormalTok{macac[,}\OperatorTok{-}\DecValTok{3}\NormalTok{]}
\NormalTok{macac}
\end{Highlighting}
\end{Shaded}

\begin{verbatim}
##    Reserva Macacos Mortes
## 1        A      22      2
## 2        A      28      7
## 3        A      37      1
## 4        A      34      2
## 5        A      13      7
## 6        A      24      4
## 7        A      39      2
## 8        A       5      4
## 9        A      33      3
## 10       A      32      9
## 11       B       7      6
## 12       B      15      6
## 13       B      12      4
## 14       B      14      1
## 15       B       4      3
## 16       B      14      1
## 17       B      16      7
## 18       B      60      2
## 19       B      13      1
## 20       B      16      8
\end{verbatim}

\begin{Shaded}
\begin{Highlighting}[]
\CommentTok{# h) Selecione os elementos da reserva A e armazene em outro data.frame, chamado A.}
\NormalTok{A <-}\StringTok{ }\KeywordTok{data.frame}\NormalTok{(macac[macac}\OperatorTok{$}\NormalTok{Reserva}\OperatorTok{==}\StringTok{"A"}\NormalTok{,])}
\NormalTok{A}
\end{Highlighting}
\end{Shaded}

\begin{verbatim}
##    Reserva Macacos Mortes
## 1        A      22      2
## 2        A      28      7
## 3        A      37      1
## 4        A      34      2
## 5        A      13      7
## 6        A      24      4
## 7        A      39      2
## 8        A       5      4
## 9        A      33      3
## 10       A      32      9
\end{verbatim}

\begin{Shaded}
\begin{Highlighting}[]
\CommentTok{# i) Calcule a média de macacos da reserva A e a média de mortes dessa reserva.}
\KeywordTok{mean}\NormalTok{(A}\OperatorTok{$}\NormalTok{Macacos)}
\end{Highlighting}
\end{Shaded}

\begin{verbatim}
## [1] 26.7
\end{verbatim}

\begin{Shaded}
\begin{Highlighting}[]
\KeywordTok{mean}\NormalTok{(A}\OperatorTok{$}\NormalTok{Mortes)}
\end{Highlighting}
\end{Shaded}

\begin{verbatim}
## [1] 4.1
\end{verbatim}

\begin{Shaded}
\begin{Highlighting}[]
\CommentTok{# j) No data.frame A, organize os dados em ordem crescente de mortes.}
\NormalTok{Aordernado <-}\StringTok{ }\NormalTok{A[}\KeywordTok{order}\NormalTok{(A}\OperatorTok{$}\NormalTok{Mortes),]}
\NormalTok{Aordernado}
\end{Highlighting}
\end{Shaded}

\begin{verbatim}
##    Reserva Macacos Mortes
## 3        A      37      1
## 1        A      22      2
## 4        A      34      2
## 7        A      39      2
## 9        A      33      3
## 6        A      24      4
## 8        A       5      4
## 2        A      28      7
## 5        A      13      7
## 10       A      32      9
\end{verbatim}

\begin{Shaded}
\begin{Highlighting}[]
\CommentTok{# k) Separe o data.frame macac por reserva.}
\NormalTok{macacSep <-}\StringTok{ }\KeywordTok{split}\NormalTok{(macac, macac}\OperatorTok{$}\NormalTok{Reserva)}
\NormalTok{macacSep}
\end{Highlighting}
\end{Shaded}

\begin{verbatim}
## $A
##    Reserva Macacos Mortes
## 1        A      22      2
## 2        A      28      7
## 3        A      37      1
## 4        A      34      2
## 5        A      13      7
## 6        A      24      4
## 7        A      39      2
## 8        A       5      4
## 9        A      33      3
## 10       A      32      9
## 
## $B
##    Reserva Macacos Mortes
## 11       B       7      6
## 12       B      15      6
## 13       B      12      4
## 14       B      14      1
## 15       B       4      3
## 16       B      14      1
## 17       B      16      7
## 18       B      60      2
## 19       B      13      1
## 20       B      16      8
\end{verbatim}

\begin{Shaded}
\begin{Highlighting}[]
\CommentTok{# 5) Crie a matriz A com os números 2,5,2,6,1,4, com 2 linhas e 3 colunas, orientada por linhas e em}
\CommentTok{# seguida multiplique essa matriz por 3.}
\NormalTok{A <-}\StringTok{ }\KeywordTok{matrix}\NormalTok{(}\KeywordTok{c}\NormalTok{(}\DecValTok{2}\NormalTok{,}\DecValTok{5}\NormalTok{,}\DecValTok{2}\NormalTok{,}\DecValTok{6}\NormalTok{,}\DecValTok{1}\NormalTok{,}\DecValTok{4}\NormalTok{), }\DecValTok{2}\NormalTok{, }\DecValTok{3}\NormalTok{, }\DataTypeTok{byrow =} \OtherTok{TRUE}\NormalTok{)}
\NormalTok{A}
\end{Highlighting}
\end{Shaded}

\begin{verbatim}
##      [,1] [,2] [,3]
## [1,]    2    5    2
## [2,]    6    1    4
\end{verbatim}

\begin{Shaded}
\begin{Highlighting}[]
\NormalTok{A <-}\StringTok{ }\NormalTok{A }\OperatorTok{*}\StringTok{ }\DecValTok{3}
\NormalTok{A}
\end{Highlighting}
\end{Shaded}

\begin{verbatim}
##      [,1] [,2] [,3]
## [1,]    6   15    6
## [2,]   18    3   12
\end{verbatim}

\begin{Shaded}
\begin{Highlighting}[]
\CommentTok{# 6) Crie a matriz D com os números 15,18,21,27,18,10,14,5,4, com 3 linhas e 3 colunas, orientada}
\CommentTok{# por linhas e em seguida multiplique essa matriz por 2:}
\NormalTok{D <-}\StringTok{ }\KeywordTok{matrix}\NormalTok{(}\KeywordTok{c}\NormalTok{(}\DecValTok{15}\NormalTok{,}\DecValTok{18}\NormalTok{,}\DecValTok{21}\NormalTok{,}\DecValTok{27}\NormalTok{,}\DecValTok{18}\NormalTok{,}\DecValTok{10}\NormalTok{,}\DecValTok{14}\NormalTok{,}\DecValTok{5}\NormalTok{,}\DecValTok{4}\NormalTok{), }\DecValTok{3}\NormalTok{, }\DecValTok{3}\NormalTok{, }\DataTypeTok{byrow =} \OtherTok{TRUE}\NormalTok{)}
\NormalTok{D}
\end{Highlighting}
\end{Shaded}

\begin{verbatim}
##      [,1] [,2] [,3]
## [1,]   15   18   21
## [2,]   27   18   10
## [3,]   14    5    4
\end{verbatim}

\begin{Shaded}
\begin{Highlighting}[]
\NormalTok{D <-}\StringTok{ }\NormalTok{D }\OperatorTok{*}\StringTok{ }\DecValTok{2}
\NormalTok{D}
\end{Highlighting}
\end{Shaded}

\begin{verbatim}
##      [,1] [,2] [,3]
## [1,]   30   36   42
## [2,]   54   36   20
## [3,]   28   10    8
\end{verbatim}

\begin{Shaded}
\begin{Highlighting}[]
\CommentTok{# 7) Resolva a operação abaixo, criando a matriz B}
\NormalTok{B <-}\StringTok{ }\DecValTok{2}\OperatorTok{/}\DecValTok{7}\OperatorTok{*}\NormalTok{(}\KeywordTok{matrix}\NormalTok{(}\KeywordTok{c}\NormalTok{(}\DecValTok{1}\NormalTok{,}\DecValTok{2}\NormalTok{,}\DecValTok{2}\NormalTok{,}\DecValTok{4}\NormalTok{,}\DecValTok{7}\NormalTok{,}\DecValTok{6}\NormalTok{), }\DecValTok{3}\NormalTok{, }\DecValTok{2}\NormalTok{, }\DataTypeTok{byrow =} \OtherTok{TRUE}\NormalTok{) }\OperatorTok{-}\StringTok{ }\KeywordTok{matrix}\NormalTok{(}\KeywordTok{c}\NormalTok{(}\DecValTok{10}\NormalTok{,}\DecValTok{20}\NormalTok{,}\DecValTok{30}\NormalTok{,}\DecValTok{40}\NormalTok{,}\DecValTok{50}\NormalTok{,}\DecValTok{60}\NormalTok{), }\DecValTok{3}\NormalTok{, }\DecValTok{2}\NormalTok{, }\DataTypeTok{byrow =} \OtherTok{TRUE}\NormalTok{))}
\NormalTok{B}
\end{Highlighting}
\end{Shaded}

\begin{verbatim}
##            [,1]       [,2]
## [1,]  -2.571429  -5.142857
## [2,]  -8.000000 -10.285714
## [3,] -12.285714 -15.428571
\end{verbatim}

\begin{Shaded}
\begin{Highlighting}[]
\CommentTok{# 8) Multiplique a matriz B pela matriz A.}
\KeywordTok{try}\NormalTok{(B}\OperatorTok{*}\NormalTok{A)}
\end{Highlighting}
\end{Shaded}

\begin{verbatim}
## Error in B * A : non-conformable arrays
\end{verbatim}

\begin{Shaded}
\begin{Highlighting}[]
\NormalTok{B}\OperatorTok\NormalTok{A}
\end{Highlighting}
\end{Shaded}

\begin{verbatim}
##           [,1]      [,2]       [,3]
## [1,] -108.0000  -54.0000  -77.14286
## [2,] -233.1429 -150.8571 -171.42857
## [3,] -351.4286 -230.5714 -258.85714
\end{verbatim}

\begin{Shaded}
\begin{Highlighting}[]
\CommentTok{# 9) Qual a amplitude da matriz A? Explique os valores encontrados.}
\KeywordTok{range}\NormalTok{(A)}
\end{Highlighting}
\end{Shaded}

\begin{verbatim}
## [1]  3 18
\end{verbatim}

\begin{Shaded}
\begin{Highlighting}[]
  \CommentTok{# O menor valor é 3 e o maior valor 18, esta é a amplitude dos valores da matriz A}

\CommentTok{# 10) Qual a soma dos elementos de A?}
\KeywordTok{sum}\NormalTok{(A)}
\end{Highlighting}
\end{Shaded}

\begin{verbatim}
## [1] 60
\end{verbatim}

\begin{Shaded}
\begin{Highlighting}[]
\CommentTok{# 11) Qual o produto dos elementos de A?}
\KeywordTok{prod}\NormalTok{(A)}
\end{Highlighting}
\end{Shaded}

\begin{verbatim}
## [1] 349920
\end{verbatim}

\begin{Shaded}
\begin{Highlighting}[]
\CommentTok{# 12) Qual é a soma das colunas de A?}
\KeywordTok{colSums}\NormalTok{(A)}
\end{Highlighting}
\end{Shaded}

\begin{verbatim}
## [1] 24 18 18
\end{verbatim}

\begin{Shaded}
\begin{Highlighting}[]
\CommentTok{# 13) Qual a soma dos valores menores que 4 na matriz B?}
\KeywordTok{sum}\NormalTok{(B[B}\OperatorTok{<}\DecValTok{4}\NormalTok{])}
\end{Highlighting}
\end{Shaded}

\begin{verbatim}
## [1] -53.71429
\end{verbatim}

\begin{Shaded}
\begin{Highlighting}[]
\CommentTok{# 14) Importe a planilha do Excel Exercicios através do comando Import Dataset. Selecione as}
\CommentTok{# colunas Idade e NF para “numérico” já ao importar o arquivo.}

\KeywordTok{library}\NormalTok{(readxl)}
\NormalTok{Exercicios <-}\StringTok{ }\KeywordTok{read_excel}\NormalTok{(}\StringTok{"Exercicios.xlsx"}\NormalTok{)}

\CommentTok{# a) verifique se é um data.frame}
\KeywordTok{is.data.frame}\NormalTok{(Exercicios)}
\end{Highlighting}
\end{Shaded}

\begin{verbatim}
## [1] TRUE
\end{verbatim}

\begin{Shaded}
\begin{Highlighting}[]
\CommentTok{# b) Transforme a coluna Sexo em fator.}
\NormalTok{Exercicios}\OperatorTok{$}\NormalTok{Sexo <-}\StringTok{ }\KeywordTok{as.factor}\NormalTok{(Exercicios}\OperatorTok{$}\NormalTok{Sexo)}
\NormalTok{Exercicios}
\end{Highlighting}
\end{Shaded}

\begin{verbatim}
## # A tibble: 6 x 4
##   Nome           Idade Sexo     NF
##   <chr>          <dbl> <fct> <dbl>
## 1 José Santos       17 M        92
## 2 Angela Dias       17 F        75
## 3 Aline Souza       16 F        81
## 4 Mayara Costa      15 F        87
## 5 Lara Lins         15 F        90
## 6 Nicolas Barros    13 M        88
\end{verbatim}

\begin{Shaded}
\begin{Highlighting}[]
\CommentTok{# c) calcule a média de Idade.}
\KeywordTok{mean}\NormalTok{(Exercicios}\OperatorTok{$}\NormalTok{Idade)}
\end{Highlighting}
\end{Shaded}

\begin{verbatim}
## [1] 15.5
\end{verbatim}

\begin{Shaded}
\begin{Highlighting}[]
\CommentTok{# 15) Crie uma função em que você tenha como input um número n e que retorne como resultado o}
\CommentTok{# quadrado de n.}
\NormalTok{fnQuadrado <-}\StringTok{ }\ControlFlowTok{function}\NormalTok{(numero) }
\NormalTok{\{}
  \KeywordTok{return}\NormalTok{(numero}\OperatorTok{^}\DecValTok{2}\NormalTok{)}
\NormalTok{\}}
\KeywordTok{fnQuadrado}\NormalTok{(}\DecValTok{4}\NormalTok{)}
\end{Highlighting}
\end{Shaded}

\begin{verbatim}
## [1] 16
\end{verbatim}

\begin{Shaded}
\begin{Highlighting}[]
\CommentTok{# 16) Crie uma função cujo input dado sejam dois números, n e m, e que lhe retorne a soma e o}
\CommentTok{# produto dos dois números. Faça com que essa função tenha m=1 caso o usuário não atribua}
\CommentTok{# nenhum valor a n:}
\NormalTok{fnSomaProduto <-}\StringTok{ }\ControlFlowTok{function}\NormalTok{(n,}\DataTypeTok{m =} \DecValTok{1}\NormalTok{) }
\NormalTok{\{}
  \KeywordTok{print}\NormalTok{(n}\OperatorTok{+}\NormalTok{m)}
  \KeywordTok{print}\NormalTok{(}\KeywordTok{prod}\NormalTok{(n,m))}
\NormalTok{\}}
\KeywordTok{fnSomaProduto}\NormalTok{(}\DecValTok{2}\NormalTok{,}\DecValTok{1}\NormalTok{)}
\end{Highlighting}
\end{Shaded}

\begin{verbatim}
## [1] 3
## [1] 2
\end{verbatim}

\begin{Shaded}
\begin{Highlighting}[]
\KeywordTok{fnSomaProduto}\NormalTok{(}\DecValTok{2}\NormalTok{)}
\end{Highlighting}
\end{Shaded}

\begin{verbatim}
## [1] 3
## [1] 2
\end{verbatim}

\begin{Shaded}
\begin{Highlighting}[]
\CommentTok{# 17) Crie uma função que subtraia a raiz quadrada de dois números e chame-a de questao3. Após}
\CommentTok{# isso teste a função criada com os valores 100 e 25.}
\NormalTok{questao3 <-}\StringTok{ }\ControlFlowTok{function}\NormalTok{(n1, n2)}
\NormalTok{\{}
  \KeywordTok{return}\NormalTok{(}\KeywordTok{sqrt}\NormalTok{(n1)}\OperatorTok{-}\KeywordTok{sqrt}\NormalTok{(n2))}
\NormalTok{\}}
\KeywordTok{questao3}\NormalTok{(}\DecValTok{100}\NormalTok{,}\DecValTok{25}\NormalTok{)}
\end{Highlighting}
\end{Shaded}

\begin{verbatim}
## [1] 5
\end{verbatim}

\begin{Shaded}
\begin{Highlighting}[]
\CommentTok{# 18) Crie uma função chamada jogar.dado, que corresponda ao jogo de um dado, com reposição.}
\CommentTok{# Depois teste para 4 jogadas e 6 jogadas desse dado.}
\NormalTok{jogar.dado <-}\StringTok{ }\ControlFlowTok{function}\NormalTok{(qtdJogadas)}
\NormalTok{\{}
  \KeywordTok{sample}\NormalTok{(}\KeywordTok{c}\NormalTok{(}\StringTok{'1'}\NormalTok{, }\StringTok{'2'}\NormalTok{, }\StringTok{'3'}\NormalTok{, }\StringTok{'4'}\NormalTok{, }\StringTok{'5'}\NormalTok{, }\StringTok{'6'}\NormalTok{), qtdJogadas, }\DataTypeTok{replace =} \OtherTok{TRUE}\NormalTok{)}
\NormalTok{\}}
\KeywordTok{jogar.dado}\NormalTok{(}\DecValTok{4}\NormalTok{)}
\end{Highlighting}
\end{Shaded}

\begin{verbatim}
## [1] "1" "3" "2" "4"
\end{verbatim}

\begin{Shaded}
\begin{Highlighting}[]
\KeywordTok{jogar.dado}\NormalTok{(}\DecValTok{6}\NormalTok{)}
\end{Highlighting}
\end{Shaded}

\begin{verbatim}
## [1] "1" "3" "6" "3" "4" "6"
\end{verbatim}

\begin{Shaded}
\begin{Highlighting}[]
\CommentTok{# 19) Foram levantadas as idades de pessoas que frequentam uma determinada lanchonete que}
\CommentTok{# vende bebidas alcoólicas. Os dados foram, 17,25,18,12,14,53,45,10,62,13,16,19. Considerando}
\CommentTok{# que as pessoas que possuem acima de 18 anos são “maiores” e as pessoas que possuem abaixo}
\CommentTok{# de 18 anos são “menores”, crie uma função chamada idades, envolvendo a condição ifelse para}
\CommentTok{# este caso e descubra quais são maiores e menores de idade.}
\NormalTok{IdadesPessoas <-}\StringTok{ }\KeywordTok{c}\NormalTok{(}\DecValTok{17}\NormalTok{,}\DecValTok{25}\NormalTok{,}\DecValTok{18}\NormalTok{,}\DecValTok{12}\NormalTok{,}\DecValTok{14}\NormalTok{,}\DecValTok{53}\NormalTok{,}\DecValTok{45}\NormalTok{,}\DecValTok{10}\NormalTok{,}\DecValTok{62}\NormalTok{,}\DecValTok{13}\NormalTok{,}\DecValTok{16}\NormalTok{,}\DecValTok{19}\NormalTok{)}
\NormalTok{idades <-}\StringTok{ }\ControlFlowTok{function}\NormalTok{(pessoas)}
\NormalTok{\{}
\NormalTok{  Maioridade <-}\StringTok{ }\KeywordTok{c}\NormalTok{()}
  \ControlFlowTok{for}\NormalTok{ (pessoa }\ControlFlowTok{in}\NormalTok{ pessoas) \{}
    \KeywordTok{ifelse}\NormalTok{ (pessoa }\OperatorTok{>=}\StringTok{ }\DecValTok{18}\NormalTok{, ehMaior <-}\StringTok{ }\NormalTok{T, ehMaior <-}\StringTok{ }\NormalTok{F)}
\NormalTok{    Maioridade <-}\StringTok{ }\KeywordTok{c}\NormalTok{(Maioridade, ehMaior);}
\NormalTok{  \}}
  \KeywordTok{print}\NormalTok{(Maioridade)}
  \KeywordTok{return}\NormalTok{(}\KeywordTok{data.frame}\NormalTok{(}\DataTypeTok{Pessoas=}\NormalTok{pessoas, Maioridade))}
\NormalTok{\}}
\KeywordTok{idades}\NormalTok{(IdadesPessoas)}
\end{Highlighting}
\end{Shaded}

\begin{verbatim}
##  [1] FALSE  TRUE  TRUE FALSE FALSE  TRUE  TRUE FALSE  TRUE FALSE FALSE  TRUE
\end{verbatim}

\begin{verbatim}
##    Pessoas Maioridade
## 1       17      FALSE
## 2       25       TRUE
## 3       18       TRUE
## 4       12      FALSE
## 5       14      FALSE
## 6       53       TRUE
## 7       45       TRUE
## 8       10      FALSE
## 9       62       TRUE
## 10      13      FALSE
## 11      16      FALSE
## 12      19       TRUE
\end{verbatim}

\begin{Shaded}
\begin{Highlighting}[]
\CommentTok{# 20) Crie uma função que calcula a área de um triângulo, chame-a de área e teste sua função com}
\CommentTok{# os valores de 5 cm para a base e 2cm para a altura.}
\NormalTok{area <-}\StringTok{ }\ControlFlowTok{function}\NormalTok{(base, altura)}
\NormalTok{\{}
  \KeywordTok{return}\NormalTok{((base}\OperatorTok{*}\NormalTok{altura)}\OperatorTok{/}\DecValTok{2}\NormalTok{)}
\NormalTok{\}}
\KeywordTok{area}\NormalTok{(}\DecValTok{5}\NormalTok{,}\DecValTok{2}\NormalTok{)}
\end{Highlighting}
\end{Shaded}

\begin{verbatim}
## [1] 5
\end{verbatim}

\begin{Shaded}
\begin{Highlighting}[]
\CommentTok{# 21) Crie uma função para resolver equações de segundo grau, onde: se delta<0, teremos “raízes}
\CommentTok{# complexas”, ou então, teremos raiz1 e raiz2, lembrando que para calcular as raízes, o delta é b^2-}
\CommentTok{#   4*a*c. Teste a função criada para:}
\NormalTok{equacao2grau <-}\StringTok{ }\ControlFlowTok{function}\NormalTok{(a, b, c)}
\NormalTok{\{}
\NormalTok{  delta <-}\StringTok{ }\NormalTok{b}\OperatorTok{^}\DecValTok{2-4}\OperatorTok{*}\NormalTok{a}\OperatorTok{*}\NormalTok{c}
  \ControlFlowTok{if}\NormalTok{ (delta}\OperatorTok{<}\DecValTok{0}\NormalTok{)}
    \KeywordTok{print}\NormalTok{(}\StringTok{"raízes complexas"}\NormalTok{)}
  \ControlFlowTok{else}\NormalTok{ \{}
\NormalTok{    raiz1 <-}\StringTok{ }\NormalTok{(}\OperatorTok{-}\NormalTok{(b)}\OperatorTok{+}\KeywordTok{sqrt}\NormalTok{(delta))}\OperatorTok{/}\NormalTok{(}\DecValTok{2}\OperatorTok{*}\NormalTok{a)}
\NormalTok{    raiz2 <-}\StringTok{ }\NormalTok{(}\OperatorTok{-}\NormalTok{(b)}\OperatorTok{-}\KeywordTok{sqrt}\NormalTok{(delta))}\OperatorTok{/}\NormalTok{(}\DecValTok{2}\OperatorTok{*}\NormalTok{a)}
    \KeywordTok{print}\NormalTok{(raiz1)}
    \KeywordTok{print}\NormalTok{(raiz2)}
\NormalTok{  \}}
\NormalTok{\}}

\CommentTok{# a) x² + 7x + 6}
\KeywordTok{equacao2grau}\NormalTok{(}\DecValTok{1}\NormalTok{, }\DecValTok{7}\NormalTok{, }\DecValTok{6}\NormalTok{)}
\end{Highlighting}
\end{Shaded}

\begin{verbatim}
## [1] -1
## [1] -6
\end{verbatim}

\begin{Shaded}
\begin{Highlighting}[]
\CommentTok{# b) x² + 3x + 5}
\KeywordTok{equacao2grau}\NormalTok{(}\DecValTok{1}\NormalTok{, }\DecValTok{3}\NormalTok{, }\DecValTok{5}\NormalTok{)}
\end{Highlighting}
\end{Shaded}

\begin{verbatim}
## [1] "raízes complexas"
\end{verbatim}

\begin{Shaded}
\begin{Highlighting}[]
\CommentTok{# 22) Um dado foi lançado 50 vezes e foram registrados os seguintes resultados:}
\CommentTok{#}
\CommentTok{# 5 4 6 1 2 5 3 1 3 3}
\CommentTok{# 4 4 1 5 5 6 1 2 5 1}
\CommentTok{# 3 4 5 1 1 6 6 2 1 1}
\CommentTok{# 4 4 4 3 4 3 2 2 2 3}
\CommentTok{# 6 6 3 2 4 2 6 6 2 1}
\CommentTok{# }
\CommentTok{# Construa o histograma dos resultados, com intervalo fechado à esquerda. Escolha um título para o}
\CommentTok{# gráfico e para os eixos x e y e a cor das barras faça de verde.}

\NormalTok{resultadoDados <-}\StringTok{ }\KeywordTok{c}\NormalTok{(}\DecValTok{5}\NormalTok{,}\DecValTok{4}\NormalTok{,}\DecValTok{6}\NormalTok{,}\DecValTok{1}\NormalTok{,}\DecValTok{2}\NormalTok{,}\DecValTok{5}\NormalTok{,}\DecValTok{3}\NormalTok{,}\DecValTok{1}\NormalTok{,}\DecValTok{3}\NormalTok{,}\DecValTok{3}\NormalTok{, }\DecValTok{4}\NormalTok{,}\DecValTok{4}\NormalTok{,}\DecValTok{1}\NormalTok{,}\DecValTok{5}\NormalTok{,}\DecValTok{5}\NormalTok{,}\DecValTok{6}\NormalTok{,}\DecValTok{1}\NormalTok{,}\DecValTok{2}\NormalTok{,}\DecValTok{5}\NormalTok{,}\DecValTok{1}\NormalTok{, }\DecValTok{3}\NormalTok{,}\DecValTok{4}\NormalTok{,}\DecValTok{5}\NormalTok{,}\DecValTok{1}\NormalTok{,}\DecValTok{1}\NormalTok{,}\DecValTok{6}\NormalTok{,}\DecValTok{6}\NormalTok{,}\DecValTok{2}\NormalTok{,}\DecValTok{1}\NormalTok{,}\DecValTok{1}\NormalTok{, }\DecValTok{4}\NormalTok{,}\DecValTok{4}\NormalTok{,}\DecValTok{4}\NormalTok{,}\DecValTok{3}\NormalTok{,}\DecValTok{4}\NormalTok{,}\DecValTok{3}\NormalTok{,}\DecValTok{2}\NormalTok{,}\DecValTok{2}\NormalTok{,}\DecValTok{2}\NormalTok{,}\DecValTok{3}\NormalTok{, }\DecValTok{6}\NormalTok{,}\DecValTok{6}\NormalTok{,}\DecValTok{3}\NormalTok{,}\DecValTok{2}\NormalTok{,}\DecValTok{4}\NormalTok{,}\DecValTok{2}\NormalTok{,}\DecValTok{6}\NormalTok{,}\DecValTok{6}\NormalTok{,}\DecValTok{2}\NormalTok{,}\DecValTok{1}\NormalTok{)}
\NormalTok{resultadoDados}
\end{Highlighting}
\end{Shaded}

\begin{verbatim}
##  [1] 5 4 6 1 2 5 3 1 3 3 4 4 1 5 5 6 1 2 5 1 3 4 5 1 1 6 6 2 1 1 4 4 4 3 4 3 2 2
## [39] 2 3 6 6 3 2 4 2 6 6 2 1
\end{verbatim}

\begin{Shaded}
\begin{Highlighting}[]
\KeywordTok{hist}\NormalTok{(resultadoDados,}
     \DataTypeTok{main =} \StringTok{"Dados lançados"}\NormalTok{,}
     \DataTypeTok{xlab =} \StringTok{"Lado do dado"}\NormalTok{,}
     \DataTypeTok{ylab =} \StringTok{"Frequência",}
\StringTok{     col="}\NormalTok{green}\StringTok{",}
\StringTok{     right = FALSE)}
\end{Highlighting}
\end{Shaded}

\includegraphics{EXERCÍCIOS-14_3_20_files/figure-latex/unnamed-chunk-1-1.pdf}

\begin{Shaded}
\begin{Highlighting}[]
\CommentTok{# 23) Os dados seguintes representam 20 observações relativas ao índice pluviométrico em}
\CommentTok{# determinado município do Estado:}
\CommentTok{#   }
\CommentTok{# Milímetros de chuva}
\CommentTok{# 144 152 159 160}
\CommentTok{# 160 151 157 146}
\CommentTok{# 154 145 151 150}
\CommentTok{# 142 146 142 141}
\CommentTok{# 141 150 143 158}
\CommentTok{#}
\CommentTok{# Construa o histograma dos resultados, com intervalo fechado à esquerda. Escolha um título para o}
\CommentTok{# gráfico e para os eixos x e y e a cor das barras faça de amarelo. Altere os limites dos eixos x e y}
\CommentTok{# para valores à sua escolha.}

\NormalTok{dadosPluviometrico <-}\StringTok{ }\KeywordTok{c}\NormalTok{(}\DecValTok{144}\NormalTok{,}\DecValTok{152}\NormalTok{,}\DecValTok{159}\NormalTok{,}\DecValTok{160}\NormalTok{,}\DecValTok{160}\NormalTok{,}\DecValTok{151}\NormalTok{,}\DecValTok{157}\NormalTok{,}\DecValTok{146}\NormalTok{,}\DecValTok{154}\NormalTok{,}\DecValTok{145}\NormalTok{,}\DecValTok{151}\NormalTok{,}\DecValTok{150}\NormalTok{,}\DecValTok{142}\NormalTok{,}\DecValTok{146}\NormalTok{,}\DecValTok{142}\NormalTok{,}\DecValTok{141}\NormalTok{,}\DecValTok{141}\NormalTok{,}\DecValTok{150}\NormalTok{,}\DecValTok{143}\NormalTok{,}\DecValTok{158}\NormalTok{)}
\NormalTok{dadosPluviometrico}
\end{Highlighting}
\end{Shaded}

\begin{verbatim}
##  [1] 144 152 159 160 160 151 157 146 154 145 151 150 142 146 142 141 141 150 143
## [20] 158
\end{verbatim}

\begin{Shaded}
\begin{Highlighting}[]
\KeywordTok{hist}\NormalTok{(dadosPluviometrico,}
     \DataTypeTok{main =} \StringTok{"Índice Pluviométrico"}\NormalTok{,}
     \DataTypeTok{xlab =} \StringTok{"Medida"}\NormalTok{,}
     \DataTypeTok{ylab =} \StringTok{"Frequência",}
\StringTok{     col="}\NormalTok{yellow}\StringTok{",}
\StringTok{     right = F,}
\StringTok{     xlim = c(125, 175),}
\StringTok{     ylim = c(0, 25))}
\end{Highlighting}
\end{Shaded}

\includegraphics{EXERCÍCIOS-14_3_20_files/figure-latex/unnamed-chunk-1-2.pdf}

\begin{Shaded}
\begin{Highlighting}[]
\CommentTok{# 24) Os dados são referentes às temperaturas diárias do mês de maio e setembro, em Fahrenheit,}
\CommentTok{# na cidade de Nova York em 1973.}

\CommentTok{# tempm=c(67,72,74,62,56,66,65,59,61,69,74,69,66,68,58,64,66,57,68,62,59,73,61,61,57,58,57,67,8}
\CommentTok{#         1,79,76)}
\CommentTok{# temps=c(91,92,93,93,87,84,80,78,75,73,81,76,77,71,71,78,67,76,68,82,64,71,81,69,63,70,77,7}
\CommentTok{#         5,76,68)}

\NormalTok{tempm=}\KeywordTok{c}\NormalTok{(}\DecValTok{67}\NormalTok{,}\DecValTok{72}\NormalTok{,}\DecValTok{74}\NormalTok{,}\DecValTok{62}\NormalTok{,}\DecValTok{56}\NormalTok{,}\DecValTok{66}\NormalTok{,}\DecValTok{65}\NormalTok{,}\DecValTok{59}\NormalTok{,}\DecValTok{61}\NormalTok{,}\DecValTok{69}\NormalTok{,}\DecValTok{74}\NormalTok{,}\DecValTok{69}\NormalTok{,}\DecValTok{66}\NormalTok{,}\DecValTok{68}\NormalTok{,}\DecValTok{58}\NormalTok{,}\DecValTok{64}\NormalTok{,}\DecValTok{66}\NormalTok{,}\DecValTok{57}\NormalTok{,}\DecValTok{68}\NormalTok{,}\DecValTok{62}\NormalTok{,}\DecValTok{59}\NormalTok{,}\DecValTok{73}\NormalTok{,}\DecValTok{61}\NormalTok{,}\DecValTok{61}\NormalTok{,}\DecValTok{57}\NormalTok{,}\DecValTok{58}\NormalTok{,}\DecValTok{57}\NormalTok{,}\DecValTok{67}\NormalTok{,}\DecValTok{81}\NormalTok{,}\DecValTok{79}\NormalTok{,}\DecValTok{76}\NormalTok{)}
\NormalTok{temps=}\KeywordTok{c}\NormalTok{(}\DecValTok{91}\NormalTok{,}\DecValTok{92}\NormalTok{,}\DecValTok{93}\NormalTok{,}\DecValTok{93}\NormalTok{,}\DecValTok{87}\NormalTok{,}\DecValTok{84}\NormalTok{,}\DecValTok{80}\NormalTok{,}\DecValTok{78}\NormalTok{,}\DecValTok{75}\NormalTok{,}\DecValTok{73}\NormalTok{,}\DecValTok{81}\NormalTok{,}\DecValTok{76}\NormalTok{,}\DecValTok{77}\NormalTok{,}\DecValTok{71}\NormalTok{,}\DecValTok{71}\NormalTok{,}\DecValTok{78}\NormalTok{,}\DecValTok{67}\NormalTok{,}\DecValTok{76}\NormalTok{,}\DecValTok{68}\NormalTok{,}\DecValTok{82}\NormalTok{,}\DecValTok{64}\NormalTok{,}\DecValTok{71}\NormalTok{,}\DecValTok{81}\NormalTok{,}\DecValTok{69}\NormalTok{,}\DecValTok{63}\NormalTok{,}\DecValTok{70}\NormalTok{,}\DecValTok{77}\NormalTok{,}\DecValTok{75}\NormalTok{,}\DecValTok{76}\NormalTok{,}\DecValTok{68}\NormalTok{)}

\CommentTok{# a) Faça o histograma das temperaturas do mês de maio. Coloque título e linhas de sombreamento}
\CommentTok{# de densidade 30 (use density=30). Faça o eixo y ter amplitude de 1 a 10.}
\KeywordTok{hist}\NormalTok{(tempm,}
     \DataTypeTok{main =} \StringTok{"Temperaturas Maio (Nova York - 1973)"}\NormalTok{,}
     \DataTypeTok{xlab =} \StringTok{"Temperatura (Fahrenheit)"}\NormalTok{,}
     \DataTypeTok{ylab =} \StringTok{"Frequência",}
\StringTok{     density=30,}
\StringTok{     ylim = c(1, 10))}
\end{Highlighting}
\end{Shaded}

\includegraphics{EXERCÍCIOS-14_3_20_files/figure-latex/unnamed-chunk-1-3.pdf}

\begin{Shaded}
\begin{Highlighting}[]
\CommentTok{# b) Faça o histograma das temperaturas do mês de setembro. Coloque título e cor = “Violet”.}
\KeywordTok{hist}\NormalTok{(temps,}
     \DataTypeTok{main =} \StringTok{"Temperaturas Setembro (Nova York - 1973)"}\NormalTok{,}
     \DataTypeTok{xlab =} \StringTok{"Temperatura (Fahrenheit)"}\NormalTok{,}
     \DataTypeTok{ylab =} \StringTok{"Frequência",}
\StringTok{     col="}\NormalTok{violet}\StringTok{")}
\end{Highlighting}
\end{Shaded}

\includegraphics{EXERCÍCIOS-14_3_20_files/figure-latex/unnamed-chunk-1-4.pdf}

\begin{Shaded}
\begin{Highlighting}[]
\CommentTok{# c) Converta as temperaturas do mês de maio para graus Celsius através da expressão °C = (°F − 32) / 1,8.}
\CommentTok{# Faça o histograma, coloque título, verifique os limites dos eixos x e y, sombreamento de densidade 25 e cor = “dark blue”.}
\NormalTok{convFahrenheitParaCelsius <-}\StringTok{ }\ControlFlowTok{function}\NormalTok{(temp)}
\NormalTok{\{}
  \KeywordTok{return}\NormalTok{((temp }\OperatorTok{-}\StringTok{ }\DecValTok{32}\NormalTok{) }\OperatorTok{/}\StringTok{ }\FloatTok{1.8}\NormalTok{)}
\NormalTok{\}}
\NormalTok{tempmCelsius <-}\StringTok{ }\KeywordTok{c}\NormalTok{()}
\ControlFlowTok{for}\NormalTok{ (temp }\ControlFlowTok{in}\NormalTok{ tempm) \{}
\NormalTok{  tempmCelsius <-}\StringTok{ }\KeywordTok{c}\NormalTok{(tempmCelsius, }\KeywordTok{convFahrenheitParaCelsius}\NormalTok{(temp))}
\NormalTok{\}}
\NormalTok{tempmCelsius}
\end{Highlighting}
\end{Shaded}

\begin{verbatim}
##  [1] 19.44444 22.22222 23.33333 16.66667 13.33333 18.88889 18.33333 15.00000
##  [9] 16.11111 20.55556 23.33333 20.55556 18.88889 20.00000 14.44444 17.77778
## [17] 18.88889 13.88889 20.00000 16.66667 15.00000 22.77778 16.11111 16.11111
## [25] 13.88889 14.44444 13.88889 19.44444 27.22222 26.11111 24.44444
\end{verbatim}

\begin{Shaded}
\begin{Highlighting}[]
\KeywordTok{hist}\NormalTok{(tempmCelsius,}
     \DataTypeTok{main =} \StringTok{"Temperaturas Maio (Nova York - 1973)"}\NormalTok{,}
     \DataTypeTok{xlab =} \StringTok{"Temperatura (Celsius)"}\NormalTok{,}
     \DataTypeTok{col =} \StringTok{"dark blue"}\NormalTok{,}
     \DataTypeTok{ylab =} \StringTok{"Frequência",}
\StringTok{     density=30,}
\StringTok{     xlim=c(10, 30),}
\StringTok{     ylim = c(0, 10))}
\end{Highlighting}
\end{Shaded}

\includegraphics{EXERCÍCIOS-14_3_20_files/figure-latex/unnamed-chunk-1-5.pdf}

\begin{Shaded}
\begin{Highlighting}[]
\CommentTok{# 25) Um pesquisador descobriu que a relação entre horas de estudo e nota da prova na disciplina}
\CommentTok{# de Estatística de determinada universidade está regida pela equação y = 1,35x+44,59.}

\NormalTok{x =}\StringTok{ }\KeywordTok{c}\NormalTok{(}\DecValTok{0}\OperatorTok{:}\DecValTok{30}\NormalTok{)}
\NormalTok{y =}\StringTok{ }\FloatTok{1.35}\OperatorTok{*}\NormalTok{x}\FloatTok{+44.59}

\CommentTok{# a) Plote esse gráfico, com x (horas de estudo) variando de 0 a 30h e notas variando de 0 a 100, dê}
\CommentTok{# títulos para os eixos e para o gráfico, dê limites coerentes para os eixos, escolha cor e formato}
\CommentTok{# para os pontos, tamanho de traço para os eixos, tamanho dos pontos.}
\KeywordTok{plot}\NormalTok{(x,}
\NormalTok{     y,}
     \DataTypeTok{xlim =} \KeywordTok{c}\NormalTok{(}\DecValTok{0}\NormalTok{,}\DecValTok{30}\NormalTok{),}
     \DataTypeTok{ylim =} \KeywordTok{c}\NormalTok{(}\DecValTok{0}\NormalTok{,}\DecValTok{100}\NormalTok{),}
     \DataTypeTok{xlab =} \StringTok{"Horas de estudo"}\NormalTok{,}
     \DataTypeTok{ylab =} \StringTok{"Notas"}\NormalTok{,}
     \DataTypeTok{col =} \StringTok{"blue"}\NormalTok{,}
     \DataTypeTok{pch =} \DecValTok{25}\NormalTok{,}
     \DataTypeTok{tcl =} \FloatTok{0.5}\NormalTok{,}
     \DataTypeTok{cex =} \FloatTok{0.8}\NormalTok{)}
\end{Highlighting}
\end{Shaded}

\includegraphics{EXERCÍCIOS-14_3_20_files/figure-latex/unnamed-chunk-1-6.pdf}

\begin{Shaded}
\begin{Highlighting}[]
\CommentTok{# b) Analise o gráfico em termos de notas versus horas de estudo.}
  \CommentTok{# Quanto maior o tempo de estudo, maior a nota da pessoa}
\end{Highlighting}
\end{Shaded}

\end{document}
